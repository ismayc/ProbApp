\documentclass[]{article}

\usepackage{amsmath}
\usepackage{amssymb}

\begin{document}

In R, $$\mathbb{P}(X = k) = \binom{k + n - 1}{k} p^n (1-p)^k$$ for $k = 0, 1, 2, \ldots n >0$ and $0 < p \le 1$.  This represents the number of failures ($k$) which occur in a sequence of Bernoulli trials before a target number of successes is reached (size = $n$).  $$\mathbb{E} (X) = \dfrac{n(1-p)}{p} \text{ and } \mathbb{V} (X) = \dfrac{n(1-p)}{p^2}.$$

In Wackerly text, $$f(x) = \binom{x -1}{r-1} p^r (1-p)^{x-r}$$ for $x = r, r+1, r+2, \ldots$ and $0 < p \le 1$.  This represents the number of trials ($x$) needed to obtain a total of $r$ successes when each trial is independently a success with probability $p$.  $$\mathbb{E} (X) = \dfrac{r}{p} \text{ and } \mathbb{V} (X) = \dfrac{r(1-p)}{p^2}$$.  

We will use the same notation as that in R so we let $x = k$ and $r = n$.  This gives $$f(k) = \binom{k -1}{n-1} p^n (1-p)^{k-n}$$ for $k = n, n+1, n+2, \ldots$ and $0 < p \le 1$ and $$\mathbb{E} (X) = \dfrac{n}{p} \text{ and } \mathbb{V} (X) = \dfrac{r(1-p)}{p^2}.$$ 

We want to show how to get these results in the previous paragraph and show that the two formulas are equal.  The total number of trials is equal to the number of successes plus the number of failures.  This implies that we need to replace everywhere in the R formula that we see $k$ with $k-n$ and subtract $n$ from the mean.  

This gives

$$\mathbb{P}(X = k) = \binom{(k-n) + n - 1}{k-n} p^n (1-p)^{k-n} = \binom{k - 1}{k-n} p^n (1-p)^{k-n}$$.  Note that since $(k - n) + (n - 1) = k-1$, $\binom{k-1}{k-n} = \binom{k-1}{n-1}$ and so $$\mathbb{P}(X = k) = \binom{k - 1}{n - 1} p^n (1-p)^{k-n}$$ as we had before in the Wackerly text. Also $$\mathbb{E} (X) = \dfrac{n(1-p)}{p} - n = \dfrac{n - np - np}{p} = \dfrac{n}{p} \text{ and } \mathbb{V} (X) = \dfrac{n(1-p)}{p^2}$$.

\end{document}